\documentclass{article}

\usepackage{fullpage}
\usepackage{titlesec}
\usepackage{color}
\usepackage{amsmath}
\usepackage{url}
\usepackage{verbatim}
\usepackage{graphicx}
\usepackage{parskip}
\usepackage{amssymb}
\usepackage{nicefrac}
\usepackage{listings} % For displaying code
\usepackage{algorithm2e} % pseudo-code
\usepackage{mathtools}
% for inline image 
\usepackage{float}
\usepackage{array}
\newcolumntype{P}[1]{>{\centering\arraybackslash}p{#1}}
% Python
\usepackage{xcolor}
\definecolor{codegreen}{rgb}{0,0.6,0}
\definecolor{codegray}{rgb}{0.5,0.5,0.5}
\definecolor{codepurple}{rgb}{0.58,0,0.82}
\definecolor{backcolour}{rgb}{0.95,0.95,0.92}

\setcounter{secnumdepth}{4}

\titleformat{\paragraph}
{\normalfont\normalsize\bfseries}{\theparagraph}{1em}{}
\titlespacing*{\paragraph}
{0pt}{3.25ex plus 1ex minus .2ex}{1.5ex plus .2ex}

\lstdefinestyle{mystyle}{
    backgroundcolor=\color{backcolour},   
    commentstyle=\color{codegreen},
    keywordstyle=\color{magenta},
    numberstyle=\tiny\color{codegray},
    stringstyle=\color{codepurple},
    basicstyle=\ttfamily\footnotesize,
    breakatwhitespace=false,         
    breaklines=true,    
    frame = lines,             
    captionpos=b,                    
    keepspaces=true,                 
    numbers=left,                    
    numbersep=5pt,                  
    showspaces=false,                
    showstringspaces=false,
    showtabs=false,                  
    tabsize=2
}

\lstset{style=mystyle}

\def\rubric#1{\gre{Rubric: \{#1\}}}{}

% Answers

\def\ans#1{\par\gre{Answer: #1}}

% Colors
\definecolor{blu}{rgb}{0,0,1}
\def\blu#1{{\color{blu}#1}}
\definecolor{gre}{rgb}{0,.5,0}
\def\gre#1{{\color{gre}#1}}
\definecolor{red}{rgb}{1,0,0}
\def\red#1{{\color{red}#1}}
\def\norm#1{\|#1\|}

% Math
\def\R{\mathbb{R}}
\def\argmax{\mathop{\rm arg\,max}}
\def\argmin{\mathop{\rm arg\,min}}
\newcommand{\mat}[1]{\begin{bmatrix}#1\end{bmatrix}}
\newcommand{\alignStar}[1]{\begin{align*}#1\end{align*}}
\def\half{\frac 1 2}

% LaTeX
\newcommand{\fig}[2]{\includegraphics[width=#1\textwidth]{#2}}
\newcommand{\centerfig}[2]{\begin{center}\includegraphics[width=#1\textwidth]{#2}\end{center}}
\newcommand{\centerfigcap}[3]{\begin{figure}[H]
\begin{center}\includegraphics[width=#1\textwidth]{#2} \caption{#3}\end{center}
\end{figure}}
\newcommand{\matCode}[1]{\lstinputlisting[language=Matlab]{a2f/#1.m}}
\def\items#1{\begin{itemize}#1\end{itemize}}
\def\enum#1{\begin{enumerate}#1\end{enumerate}}

\begin{document}


\title{\vspace{-20mm}
CPSC 532W Assignment 2}
\author{Ali Seyfi - 97446637}
\date{}
\maketitle

\section{Evaluation Based Sampling}

\blu{Write a FOPPL evaluator that evaluates the FOPPL as described in lecture and in the book (Alg. 6)}

\subsection{Code}
\blu{Provide code snippets that document critical aspects of your implementation sufficient to allow us to quickly determine whether or not you individually completed the assignment.}
\red{Based on what was mentioned in the assignment description, I ignored the observe statements for this assignment.} 
\subsubsection{primitives}
\label{primitives}
\lstinputlisting[language=Python, caption=primitives.py - Base primitives, firstline=104, lastline=135]{../primitives.py}
\pagebreak
\lstinputlisting[language=Python, caption=primitives.py - Functions, firstline=49, lastline=101]{../primitives.py}
\pagebreak
\lstinputlisting[language=Python, caption=primitives.py - Distributions, firstline=1, lastline=47]{../primitives.py}
\lstinputlisting[language=Python, caption=primitives.py - distlist, firstline=137, lastline=144]{../primitives.py}
\pagebreak
\subsubsection{evaluate program}
\lstinputlisting[language=Python, caption=evaluation\_based\_sampling.py - evaluate\_program, firstline=6, lastline=32]{../evaluation_based_sampling.py}
\subsubsection{eval}
\lstinputlisting[language=Python, caption=evaluation\_based\_sampling.py - evaluate\_program, firstline=34, lastline=79]{../evaluation_based_sampling.py}
\subsection{Results}
\subsubsection{Tests Results}
\blu{Results of Tests}\\
Here are the results of the test files:\\
\textbf{Deterministic}\\
tensor(7., dtype=torch.float64)\\
Test passed\\
tensor(1.4142, dtype=torch.float64)\\
Test passed\\
tensor(24., dtype=torch.float64)\\
Test passed\\
tensor(0.2500, dtype=torch.float64)\\
Test passed\\
tensor(0.1802, dtype=torch.float64)\\
Test passed\\
tensor([2., 3., 4., 5.], dtype=torch.float64)\\
Test passed\\
tensor(4., dtype=torch.float64)\\
Test passed\\
tensor([2., 3., 3., 5.], dtype=torch.float64)\\
Test passed\\
tensor(2., dtype=torch.float64)\\
Test passed\\
tensor(5., dtype=torch.float64)\\
Test passed\\
tensor([2.0000, 3.0000, 4.0000, 5.0000, 3.1400], dtype=torch.float64)\\
Test passed\\
tensor(5.3000, dtype=torch.float64)\\
Test passed\\
{6.0: tensor(2., dtype=torch.float64), 1.0: tensor(3.2000, dtype=torch.float64)}\\
Test passed\\
All deterministic tests passed\\
\pagebreak

\textbf{Probabilistic}
('normal', 5, 1.4142136)\\
p value 0.0012557687750447137\\
Test 1 passed\\
('beta', 2.0, 5.0)\\
p value 0.7075526288252392\\
Test 2 passed\\
('exponential', 0.0, 5.0)\\
p value 0.09598741374360209\\
Test 3 passed\\
('normal', 5.3, 3.2)\\
p value 0.016072128150235995\\
Test 4 passed\\
('normalmix', 0.1, -1, 0.3, 0.9, 1, 0.3)\\
p value 0.3676183768222948\\
Test 5 passed\\
('normal', 0, 1.44)\\
p value 0.1907566727266461\\
Test 6 passed\\
All probabilistic tests passed\\
\subsubsection{Marginal Expectation}
\blu{Draw 1000 samples for each of the task programs and report marginal expectations for all return value dimensions.}\\
Definitely  the result changes each time we draw the samples, and in the following I have provided the result of the only one run of the code.
\paragraph{Task 1}
Marginal Expectation of mu is: \textbf{0.9896}
\paragraph{Task 2}
Marginal Expectation of slope is: \textbf{-0.1806}\\
Marginal Expectation of bias is: \textbf{0.0168}
\paragraph{Task 3}
Marginal Expectation of each step is these numbers by order:\\ \textbf{1.8824,1.6470,1.7058,1.6471,1.7059,0.9412,1.1176,1.2941,\\1.2353,1.7647,1.5294,1.11765,1.11765,1.3529,1.3529,1.5882}
\paragraph{Task 4}
Marginal Expectation of $w_1$ elements are:
\[
\begin{bmatrix}
-0.1264 \\ -0.3671 \\ 0.0419 \\ 0.0767 \\ 0.1846 \\ -0.1250 \\ 0.2340 \\ 0.2118 \\ -0.1531 \\ 0.6157 \\ -0.1788 \\ 0.1203 \\ 0.8325 \\ -0.1807 \\ 0.0838 \\ -0.2090 \\ -0.0974
\end{bmatrix}
\]

Marginal Expectation of $b_1$ elements are:
\[
\begin{bmatrix}
0.2112 \\ 0.5417 \\ 0.048 \\ 0.2376 \\ -0.0978 \\ -0.0586 \\ 0.3023 \\ -0.4403 \\ -0.5109 \\
-0.0517\\
-0.1594\\
-0.1185\\
0.7249\\
0.0214\\
0.0119\\
-0.3293\\
0.0979
\end{bmatrix}
\]

Marginal Expectation of $w_2$ elements are:
\[
\begin{bmatrix}
-0.0354 &
0.0195&
-0.0046&
-0.0041&
-0.0542&
0.028&
-0.0106&
0.0013&
0.0318&
0.0063\\
0.0385&
-0.0414&
-0.0328&
-0.0524&
0.0058&
-0.0672&
-0.0242&
-0.0182&
-0.0357&
0.0026\\
-0.0005&
0.0281&
-0.0467&
-0.0054&
-0.0168&
0.0125&
-0.0054&
0.0226&
0.0022&
-0.023\\
-0.0065&
0.0267&
0.03&
0.0366&
0.0354&
0.046&
0.0115&
-0.0267&
0.0344&
-0.004\\
0.0507&
-0.0629&
0.0439&
0.0585&
0.0141&
-0.0153&
-0.0057&
0.0291&
-0.0244&
0.0308\\
-0.0106&
-0.0203&
0.0332&
0.0199&
0.0279&
-0.0208&
0.0066&
0.0153&
0.0396&
-0.0323\\
-0.0101&
-0.0367&
0.0427&
0.0245&
-0.0546&
0.0185&
0.0037&
-0.0173&
0.0014&
0.0599\\
-0.0261&
0.0029&
-0.0164&
-0.0138&
0.0364&
0.0285&
0.0475&
-0.0012&
-0.0091&
-0.052\\
-0.0146&
-0.0539&
-0.0297&
-0.0349&
0.0144&
-0.0121&
-0.0167&
-0.0114&
0.0126&
0.0041\\
0.0027&
-0.0295&
-0.0509&
0.0139&
0.0441&
0.0168&
-0.0433&
-0.0706&
0.0254&
0.0179

\end{bmatrix}
\]

Marginal Expectation of $b_2$ elements are:
\[
\begin{bmatrix}
-0.3279\\
0.2553\\
0.4049\\
0.1395\\
0.1667\\
-0.3218\\
0.3551\\
0.1866\\
0.1888\\
0.0709\\
-0.0666\\
-0.0663\\
0.0215\\
0.3254\\
0.198\\
-0.1191\\
-0.1661
\end{bmatrix}
\]
\subsubsection{Histograms}
\paragraph{Task 1}
\centerfigcap{0.6}{../Results/eval_1}{Histogram of mu}
\paragraph{Task 2}
\centerfigcap{0.6}{../Results/eval_2_slope}{Histogram of slope}
\centerfigcap{0.6}{../Results/eval_2_bias}{Histogram of bias}
\centerfigcap{0.7}{../Results/eval_2_both}{Histogram of slope and bias together}
\paragraph{Task 3}
\centerfigcap{1}{../Results/eval_3}{Histogram of states in each step}
\paragraph{Task 4}
\centerfigcap{0.7}{../Results/eval_4_w1}{Histogram of $W_1$ elements together}
\centerfigcap{0.7}{../Results/eval_4_b1}{Histogram of $b_1$ elements together}
\centerfigcap{1}{../Results/eval_4_w2}{Histogram of $W_2$ elements together}
\centerfigcap{0.7}{../Results/eval_4_b2}{Histogram of $b_2$ elements together}
\pagebreak
\section{Graph Based Sampling}
\blu{Write a FOPPL evaluator that evaluates the FOPPL program via ancestral sampling in the graphical model.}

\subsection{Code}
\blu{Provide code snippets that document critical aspects of your implementation sufficient to allow us to quickly determine whether or not you individually completed the assignment.}
\subsubsection{primitives}
This part is exactly same as \ref{primitives}
\subsubsection{topological sort}
\lstinputlisting[language=Python, caption=graph\_based\_sampling.py - topological\_sort, firstline=11, lastline=33]{../graph_based_sampling.py}
\subsubsection{environment}
\lstinputlisting[language=Python, caption=graph\_based\_sampling.py - environment, firstline=37, lastline=37]{../graph_based_sampling.py}
\subsubsection{deterministic eval}
\lstinputlisting[language=Python, caption=graph\_based\_sampling.py - deterministic\_eval, firstline=39, lastline=44]{../graph_based_sampling.py}
\subsubsection{value substitution}
\lstinputlisting[language=Python, caption=graph\_based\_sampling.py - value\_subs, firstline=46, lastline=56]{../graph_based_sampling.py}
\subsubsection{sample from joint}
\lstinputlisting[language=Python, caption=graph\_based\_sampling.py - sample\_from\_joint, firstline=58, lastline=70]{../graph_based_sampling.py}

\subsection{Results}
\subsubsection{Tests Results}
\blu{Results of Tests}\\
Here are the results of the test files:\\
\textbf{Deterministic}\\
tensor(7., dtype=torch.float64)\\
Test passed\\
tensor(1.4142, dtype=torch.float64)\\
Test passed\\
tensor(24., dtype=torch.float64)\\
Test passed\\
tensor(0.2500, dtype=torch.float64)\\
Test passed\\
tensor(0.1802, dtype=torch.float64)\\
Test passed\\
tensor([2., 3., 4., 5.], dtype=torch.float64)\\
Test passed\\
tensor(4., dtype=torch.float64)\\
Test passed\\
tensor([2., 3., 3., 5.], dtype=torch.float64)\\
Test passed\\
tensor(2., dtype=torch.float64)\\
Test passed\\
tensor(5., dtype=torch.float64)\\
Test passed\\
tensor([2.0000, 3.0000, 4.0000, 5.0000, 3.1400], dtype=torch.float64)\\
Test passed\\
tensor(5.3000, dtype=torch.float64)\\
Test passed\\
All deterministic tests passed\\
\textbf{Probabilistic}
('normal', 5, 1.4142136)\\
p value 0.5663323302205264\\
('beta', 2.0, 5.0)\\
p value 0.3206785223599047\\
('exponential', 0.0, 5.0)\\
p value 0.32984046631598296\\
('normal', 5.3, 3.2)\\
p value 0.5410338117003856\\
('normalmix', 0.1, -1, 0.3, 0.9, 1, 0.3)\\
p value 0.03844514904336682\\
('normal', 0, 1.44)\\
p value 0.6072736562958746\\
All probabilistic tests passed
\subsubsection{Marginal Expectation}
\blu{Draw 1000 samples for each of the task programs and report marginal expectations for all return value dimensions.}\\
Definitely  the result changes each time we draw the samples, and in the following I have provided the result of the only one run of the code.
\paragraph{Task 1}
Marginal Expectation of mu is: \textbf{1.0183}
\paragraph{Task 2}
Marginal Expectation of slope is: \textbf{-0.1693}\\
Marginal Expectation of bias is: \textbf{0.1126}
\paragraph{Task 3}
Marginal Expectation of each step is these numbers by order:\\ \textbf{1.2353
1.3529
1.5294
1.5882
0.8824
1.3529
1.9412
1.3529\\
1.4118
1.5294
1.6471
1.2353
1.2941
1.1176
1.2941
1.5294
1.2353}
\paragraph{Task 4}
Marginal Expectation of $w_1$ elements are:
\[
\begin{bmatrix}
-0.0116\\
-0.1254\\
0.3629\\
-0.2237\\
0.2997\\
-0.0427\\
-0.0172\\
-0.2322\\
0.0637\\
-0.2389\\
0.172\\
-0.1198\\
0.2513\\
-0.3007\\
-0.1563\\
-0.0328\\
-0.1635
\end{bmatrix}
\]

Marginal Expectation of $b_1$ elements are:
\[
\begin{bmatrix}
-0.213\\
0.1301\\
-0.3042\\
0.1645\\
-0.2951\\
-0.0824\\
0.5833\\
-0.1653\\
0.1151\\
-0.1317\\
0.1903\\
-0.1795\\
-0.4252\\
0.7888\\
-0.0638\\
0.3291\\
-0.2595
\end{bmatrix}
\]

Marginal Expectation of $w_2$ elements are:
\[
\begin{bmatrix}
-0.0142&
-0.0052&
-0.0494&
0.0185&
-0.0629&
-0.0547&
0.009&
-0.0188&
0.0162&
-0.004\\
-0.0064&
0.0193&
-0.0407&
0.0317&
0.0051&
-0.0126&
-0.0207&
-0.0025&
0.0251&
-0.0516\\
-0.057&
-0.0017&
0.0374&
0.0038&
-0.0176&
-0.0082&
0.0151&
-0.0269&
-0.0085&
-0.0243\\
0.0122&
-0.0005&
-0.0448&
-0.0564&
0.0005&
0.0345&
0.0718&
0.0308&
0.005&
-0.0089\\
-0.0819&
-0.0145&
-0.0286&
-0.0325&
-0.0565&
0.044&
0.0178&
-0.0043&
0.0271&
0.0003\\
0.0219&
0.0345&
0.0251&
0.0254&
-0.0139&
0.047&
0.0051&
0.0255&
0.0469&
0.0204\\
0.0492&
-0.0257&
-0.003&
0.0182&
-0.035&
-0.0264&
0.0609&
0.0106&
0.0003&
-0.0122\\
-0.019&
0.0187&
-0.0308&
-0.0158&
0.0189&
-0.0126&
0.0241&
-0.0395&
-0.0088&
0.0434\\
-0.0166&
-0.0015&
-0.0674&
0.0265&
-0.0046&
0.0763&
-0.0201&
0.0072&
0.0231&
-0.004\\
-0.0346&
-0.026&
0.0248&
-0.0126&
-0.0444&
0.0066&
0.0662&
-0.0013&
-0.0439&
0.0088
\end{bmatrix}
\]

Marginal Expectation of $b_2$ elements are:
\[
\begin{bmatrix}
0.0754\\
-0.044\\
0.1044\\
-0.3383\\
0.115\\
-0.0623\\
-0.6844\\
0.255\\
0.0021\\
0.3057\\
1.155\\
0.1589\\
0.0446\\
-0.1136\\
-1.0082\\
-0.3659\\
-0.0531
\end{bmatrix}
\]
\subsubsection{Histograms}
\paragraph{Task 1}
\centerfigcap{0.6}{../Results/graph_1}{Histogram of mu}
\paragraph{Task 2}
\centerfigcap{0.6}{../Results/graph_2_slope}{Histogram of slope}
\centerfigcap{0.6}{../Results/graph_2_bias}{Histogram of bias}
\centerfigcap{0.7}{../Results/graph_2_both}{Histogram of slope and bias together}
\paragraph{Task 3}
\centerfigcap{1}{../Results/graph_3}{Histogram of states in each step}
\paragraph{Task 4}
\centerfigcap{0.7}{../Results/graph_4_w1}{Histogram of $W_1$ elements together}
\centerfigcap{0.7}{../Results/graph_4_b1}{Histogram of $b_1$ elements together}
\centerfigcap{1}{../Results/graph_4_w2}{Histogram of $W_2$ elements together}
\centerfigcap{0.7}{../Results/graph_4_b2}{Histogram of $b_2$ elements together}
\end{document}
%