\documentclass{article}

\usepackage{fullpage}
\usepackage{titlesec}
\usepackage{color}
\usepackage{amsmath}
\usepackage{url}
\usepackage{verbatim}
\usepackage{graphicx}
\usepackage{parskip}
\usepackage{amssymb}
\usepackage{nicefrac}
\usepackage{listings} % For displaying code
\usepackage{algorithm2e} % pseudo-code
\usepackage{mathtools}
% for inline image 
\usepackage{float}
\usepackage{array}
\newcolumntype{P}[1]{>{\centering\arraybackslash}p{#1}}
% Python
\usepackage{xcolor}
\definecolor{codegreen}{rgb}{0,0.6,0}
\definecolor{codegray}{rgb}{0.5,0.5,0.5}
\definecolor{codepurple}{rgb}{0.58,0,0.82}
\definecolor{backcolour}{rgb}{0.95,0.95,0.92}

\setcounter{secnumdepth}{4}

\titleformat{\paragraph}
{\normalfont\normalsize\bfseries}{\theparagraph}{1em}{}
\titlespacing*{\paragraph}
{0pt}{3.25ex plus 1ex minus .2ex}{1.5ex plus .2ex}

\lstdefinestyle{mystyle}{
    backgroundcolor=\color{backcolour},   
    commentstyle=\color{codegreen},
    keywordstyle=\color{magenta},
    numberstyle=\tiny\color{codegray},
    stringstyle=\color{codepurple},
    basicstyle=\ttfamily\footnotesize,
    breakatwhitespace=false,         
    breaklines=true,    
    frame = lines,             
    captionpos=b,                    
    keepspaces=true,                 
    numbers=left,                    
    numbersep=5pt,                  
    showspaces=false,                
    showstringspaces=false,
    showtabs=false,                  
    tabsize=2
}

\lstset{style=mystyle}

\def\rubric#1{\gre{Rubric: \{#1\}}}{}

% Answers

\def\ans#1{\par\gre{Answer: #1}}

% Colors
\definecolor{blu}{rgb}{0,0,1}
\def\blu#1{{\color{blu}#1}}
\definecolor{gre}{rgb}{0,.5,0}
\def\gre#1{{\color{gre}#1}}
\definecolor{red}{rgb}{1,0,0}
\def\red#1{{\color{red}#1}}
\def\norm#1{\|#1\|}

% Math
\def\R{\mathbb{R}}
\def\argmax{\mathop{\rm arg\,max}}
\def\argmin{\mathop{\rm arg\,min}}
\newcommand{\mat}[1]{\begin{bmatrix}#1\end{bmatrix}}
\newcommand{\alignStar}[1]{\begin{align*}#1\end{align*}}
\def\half{\frac 1 2}

% LaTeX
\newcommand{\fig}[2]{\includegraphics[width=#1\textwidth]{#2}}
\newcommand{\centerfig}[2]{\begin{center}\includegraphics[width=#1\textwidth]{#2}\end{center}}
\newcommand{\centerfigcap}[3]{\begin{figure}[H]
\begin{center}\includegraphics[width=#1\textwidth]{#2} \caption{#3}\end{center}
\end{figure}}
\newcommand{\matCode}[1]{\lstinputlisting[language=Matlab]{a2f/#1.m}}
\def\items#1{\begin{itemize}#1\end{itemize}}
\def\enum#1{\begin{enumerate}#1\end{enumerate}}

\begin{document}


\title{\vspace{-20mm}
CPSC 532W Assignment 3}
\author{Ali Seyfi - 97446637}
\date{}
\maketitle

Here is the link to the repository:\\
\url{https://github.com/aliseyfi75/Probabilistic-Programming/tree/master/Assignment_3}

\section{Importance Sampling}

\blu{Write IS sampler that consume the output produced by the Daphne compiler and the evaluators you wrote in completing HW 2. }

\subsection{Code}
\blu{Provide code snippets that document critical aspects of your implementation sufficient to allow us to quickly determine whether or not you individually completed the assignment.}
\subsubsection{primitives}
\label{primitives}
\lstinputlisting[language=Python, caption=primitives.py - Base primitives, firstline=124, lastline=158]{../primitives.py}
\pagebreak
\lstinputlisting[language=Python, caption=primitives.py - Functions, firstline=69, lastline=121]{../primitives.py}
\pagebreak
\lstinputlisting[language=Python, caption=primitives.py - Distributions, firstline=5, lastline=66]{../primitives.py}
\lstinputlisting[language=Python, caption=primitives.py - distlist, firstline=160, lastline=171]{../primitives.py}
\subsubsection{evaluate program}
\lstinputlisting[language=Python, caption=evaluation\_based\_sampling.py - evaluate\_program, firstline=12, lastline=39]{../evaluation_based_sampling.py}
\pagebreak
\subsubsection{eval}
\lstinputlisting[language=Python, caption=evaluation\_based\_sampling.py - evaluate\_program, firstline=41, lastline=93]{../evaluation_based_sampling.py}
\pagebreak
\subsubsection{likelihood weighting}
\lstinputlisting[language=Python, caption=evaluation\_based\_sampling.py - likelihood\_weighting, firstline=95, lastline=108]{../evaluation_based_sampling.py}
\subsubsection{expectation calculator}
\lstinputlisting[language=Python, caption=evaluation\_based\_sampling.py - expectation\_calculator, firstline=110, lastline=113]{../evaluation_based_sampling.py}
\subsection{Results}

I draw $10^5$ samples for each task and the results are in the following:
\paragraph{Task 1}
Time of drawing samples: \textbf{16.51 seconds}\\
Posterior mean of mu is: \textbf{7.2514}\\
Posterior variance of mu is: \textbf{0.8652}
\paragraph{Task 2}
Time of drawing samples: \textbf{145.30 seconds}\\
Posterior mean of slope is: \textbf{1.9222}\\
Posterior variance of slope is: \textbf{0.0237}\\
Posterior mean of bias is: \textbf{0.9856}\\
Posterior variance of bias is: \textbf{0.6657}\\
Posterior covariance matrix of slope and bias:
$
\begin{bmatrix}
3.6949 & 1.8946\\
1.8946 & 0.9715
\end{bmatrix}
$
\paragraph{Task 3}
Time of drawing samples: \textbf{94.24 seconds}\\
Posterior mean of probability that the first and second datapoint are in the same cluster is: \textbf{0.7517}\\
Posterior variance of probability that the first and second datapoint are in the same cluster is: \textbf{0.1866}
\paragraph{Task 4}
Time of drawing samples: \textbf{31.27 seconds}\\
Posterior mean of probability that it is raining: \textbf{0.3195}\\
Posterior variance of probability that it is raining: \textbf{0.2174}
\paragraph{Task 5}
Time of drawing samples: \textbf{17.77 seconds}\\
Posterior marginal mean of x is: \textbf{4.0185}\\
Posterior marginal variance of x is: \textbf{0.4771}\\
Posterior marginal mean of y is: \textbf{2.9814}\\
Posterior marginal variance of y is: \textbf{0.4771}
\subsubsection{Histograms}
\paragraph{Task 1}
\centerfigcap{0.6}{../figures/Importance_Sampling_plt_hist_program_1_d_0}{Histogram of posterior distribution of mu}
\paragraph{Task 2}

\begin{minipage}{.5\textwidth}
  \centering
  \centerfigcap{1}{../figures/Importance_Sampling_plt_hist_program_2_d_0}{Histogram of posterior distribution of slope}
\end{minipage}%
\begin{minipage}{.5\textwidth}
  \centering
  \centerfigcap{1}{../figures/Importance_Sampling_plt_hist_program_2_d_1}{Histogram of posterior distribution of bias}
\end{minipage}

\paragraph{Task 3}
\centerfigcap{0.75}{../figures/Importance_Sampling_plt_hist_program_3_d_0}{Histogram of posterior distribution of being in same cluster}
\paragraph{Task 4}
\centerfigcap{0.7}{../figures/Importance_Sampling_plt_hist_program_4_d_0}{Histogram of posterior distribution of is\_raining}
\paragraph{Task 5}

\begin{minipage}{.5\textwidth}
  \centering
\centerfigcap{1}{../figures/Importance_Sampling_plt_hist_program_5_d_0}{Histogram of posterior distribution of x}
\end{minipage}%
\begin{minipage}{.5\textwidth}
  \centering
\centerfigcap{1}{../figures/Importance_Sampling_plt_hist_program_5_d_1}{Histogram of posterior distribution of y}
\end{minipage}

\section{Trick of task 5}

In order to get a good answer in this task, I have approximated the Dirac distribution with a normal distribution with the mean equal to the center of Dirac distribution with a really low variance (in case of Importance Sampling and MH within Gibbs,  $10^{-5}$ gave me good results, and for HMC I got good results with variance equal to $10^{-3}$).
\pagebreak
\section{Graphical based sampling code}
\subsubsection{primitives}
\label{primitives}
Primitives are same as \ref{primitives}
\subsubsection{topological sort}
\lstinputlisting[language=Python, caption=graph\_based\_sampling.py - topological\_sort, firstline=15, lastline=37]{../graph_based_sampling.py}
\subsubsection{environment}
\lstinputlisting[language=Python, caption=graph\_based\_sampling.py - environment, firstline=41, lastline=41]{../graph_based_sampling.py}
\subsubsection{deterministic eval}
\lstinputlisting[language=Python, caption=graph\_based\_sampling.py - deterministic\_eval, firstline=43, lastline=48]{../graph_based_sampling.py}
\pagebreak
\subsubsection{value substitution}
\lstinputlisting[language=Python, caption=graph\_based\_sampling.py - value\_subs, firstline=50, lastline=60]{../graph_based_sampling.py}
\subsubsection{sample from joint}
\lstinputlisting[language=Python, caption=graph\_based\_sampling.py - sample\_from\_joint, firstline=62, lastline=78]{../graph_based_sampling.py}
\pagebreak
\section{MH within Gibbs}

\blu{Write MH within Gibbs sampler that consume the output produced by the Daphne compiler and the evaluators you wrote in completing HW 2. }

\subsection{Code}
\blu{Provide code snippets that document critical aspects of your implementation sufficient to allow us to quickly determine whether or not you individually completed the assignment.}
\subsubsection{MH within Gibbs sampling}
\lstinputlisting[language=Python, caption=graph\_based\_sampling.py - mh\_within\_gibbs\_sampling, firstline=82, lastline=100]{../graph_based_sampling.py}
\subsubsection{extract variables}
\lstinputlisting[language=Python, caption=graph\_based\_sampling.py - extract\_variables, firstline=103, lastline=109]{../graph_based_sampling.py}
\subsubsection{extender}
\lstinputlisting[language=Python, caption=graph\_based\_sampling.py - extender, firstline=112, lastline=116]{../graph_based_sampling.py}
\subsubsection{extract free variables}
\lstinputlisting[language=Python, caption=graph\_based\_sampling.py - extract\_free\_variables, firstline=118, lastline=136]{../graph_based_sampling.py}
\subsubsection{Gibbs step}
\lstinputlisting[language=Python, caption=graph\_based\_sampling.py - Gibbs\_step, firstline=139, lastline=147]{../graph_based_sampling.py}
\subsubsection{MH accept}
\lstinputlisting[language=Python, caption=graph\_based\_sampling.py - MH\_accept, firstline=150, lastline=164]{../graph_based_sampling.py}
\subsection{Results}
I draw $10^5$ samples for each task and the results are in the following:
\paragraph{Task 1}
Time of drawing samples: \textbf{54.98 seconds}\\
Posterior mean of mu is: \textbf{7.2882}\\
Posterior variance of mu is: \textbf{0.8270}
\paragraph{Task 2}
Time of drawing samples: \textbf{220.52 seconds}\\
Posterior mean of slope is: \textbf{2.1574}\\
Posterior variance of slope is: \textbf{0.0597}\\
Posterior mean of bias is: \textbf{-0.5397}\\
Posterior variance of bias is: \textbf{0.8999}\\
Posterior covariance matrix of slope and bias:
$\begin{bmatrix}
0.0751 & -0.2611 \\
-0.2611 & 1.0883
\end{bmatrix}$
\paragraph{Task 3}
This time I draw $10^4$ samples.
Time of drawing samples: \textbf{190.16 seconds}\\
Posterior mean of probability that the first and second datapoint are in the same cluster is: \textbf{0.7508}\\
Posterior variance of probability that the first and second datapoint are in the same cluster is: \textbf{0.1871}
\paragraph{Task 4}
Time of drawing samples: \textbf{207.88 seconds}\\
Posterior mean of probability that it is raining: \textbf{0.3216}\\
Posterior variance of probability that it is raining: \textbf{0.2182}
\paragraph{Task 5}
Time of drawing samples: \textbf{84.22 seconds}\\
Posterior marginal mean of x is: \textbf{-4.0088}\\
Posterior marginal variance of x is: \textbf{3.5686e-05}\\
Posterior marginal mean of y is: \textbf{11.0087}\\
Posterior marginal variance of y is: \textbf{2.1461e-04}
\subsubsection{Histograms}
\paragraph{Task 1}
\centerfigcap{0.7}{../figures/MH_within_Gibbs_plt_hist_program_1_d_0}{Histogram of posterior distribution of mu}
\centerfigcap{0.8}{../figures/MH_within_Gibbs_plt_trace_program_1_d_0}{Sample trace plots of mu}
\centerfigcap{0.8}{../figures/MH_within_Gibbs_plt_log_joint_program_1}{Joint log likelihood}
\paragraph{Task 2}

\begin{minipage}{.5\textwidth}
  \centering
  \centerfigcap{1}{../figures/MH_within_Gibbs_plt_hist_program_2_d_0}{Histogram of posterior distribution of slope}
\end{minipage}%
\begin{minipage}{.5\textwidth}
  \centering
  \centerfigcap{1}{../figures/MH_within_Gibbs_plt_hist_program_2_d_1}{Histogram of posterior distribution of bias}
\end{minipage}

\begin{minipage}{.5\textwidth}
  \centering
  \centerfigcap{1}{../figures/MH_within_Gibbs_plt_trace_program_2_d_0}{Sample trace plots of slope}
\end{minipage}%
\begin{minipage}{.5\textwidth}
  \centering
  \centerfigcap{1}{../figures/MH_within_Gibbs_plt_trace_program_2_d_1}{Sample trace plots of bias}
\end{minipage}

\centerfigcap{1}{../figures/MH_within_Gibbs_plt_log_joint_program_2}{Joint log likelihood}


\paragraph{Task 3}
\centerfigcap{0.6}{../figures/MH_within_Gibbs_plt_hist_program_3_d_0}{Histogram of posterior distribution of being in same cluster}
\paragraph{Task 4}
\centerfigcap{0.6}{../figures/MH_within_Gibbs_plt_hist_program_4_d_0}{Histogram of posterior distribution of is\_raining}
\paragraph{Task 5}

\begin{minipage}{.5\textwidth}
  \centering
\centerfigcap{1}{../figures/MH_within_Gibbs_plt_hist_program_5_d_0}{Histogram of posterior distribution of x}
\end{minipage}%
\begin{minipage}{.5\textwidth}
  \centering
\centerfigcap{1}{../figures/MH_within_Gibbs_plt_hist_program_5_d_1}{Histogram of posterior distribution of y}
\end{minipage}

\begin{minipage}{.5\textwidth}
  \centering
  \centerfigcap{1}{../figures/MH_within_Gibbs_plt_trace_program_5_d_0}{Sample trace plots of slope}
\end{minipage}%
\begin{minipage}{.5\textwidth}
  \centering
  \centerfigcap{1}{../figures/MH_within_Gibbs_plt_trace_program_5_d_1}{Sample trace plots of bias}
\end{minipage}

\centerfigcap{1}{../figures/MH_within_Gibbs_plt_log_joint_program_5}{Joint log likelihood}

\pagebreak

\section{Hamiltonian Monte Carlo}
\subsection{Code}
\subsubsection{HMC}
\lstinputlisting[language=Python, caption=graph\_based\_sampling.py - hmc, firstline=168, lastline=218]{../graph_based_sampling.py}
\subsubsection{energy}
\lstinputlisting[language=Python, caption=graph\_based\_sampling.py - energy, firstline=220, lastline=229]{../graph_based_sampling.py}
\subsubsection{leapfrog}
\lstinputlisting[language=Python, caption=graph\_based\_sampling.py - leapfrog, firstline=231, lastline=239]{../graph_based_sampling.py}
\subsubsection{detach dictionary and add vector}
\lstinputlisting[language=Python, caption=graph\_based\_sampling.py - detach\_and\_add\_dict\_vector, firstline=241, lastline=246]{../graph_based_sampling.py}
\subsubsection{grad energy}
\lstinputlisting[language=Python, caption=graph\_based\_sampling.py - grad\_energy, firstline=248, lastline=257]{../graph_based_sampling.py}
\subsection{Results}
I draw $10^4$ samples for each task and the results are in the following:
\paragraph{Task 1}
Time of drawing samples: \textbf{34.80 seconds}\\
Posterior mean of mu is: \textbf{7.3272}\\
Posterior variance of mu is: \textbf{0.8059}
\paragraph{Task 2}
Time of drawing samples: \textbf{104.60 seconds}\\
Posterior mean of slope is: \textbf{2.1118}\\
Posterior variance of slope is: \textbf{0.1792}\\
Posterior mean of bias is: \textbf{-0.5026}\\
Posterior variance of bias is: \textbf{0.8677}\\
Posterior covariance matrix of slope and bias:
$\begin{bmatrix}
0.1792 & -0.2515 \\
-0.2515 & 0.8678
\end{bmatrix}$
\paragraph{Task 5}

Time of drawing samples: \textbf{294.33 seconds}\\
Posterior marginal mean of x is: \textbf{-8.8936}\\
Posterior marginal variance of x is: \textbf{-2.2888e-05}\\
Posterior marginal mean of y is: \textbf{13.7359}\\
Posterior marginal variance of y is: \textbf{1.0681e-04}
\subsubsection{Histograms}
\paragraph{Task 1}
\centerfigcap{0.8}{../figures/HMC_plt_hist_program_1_d_0}{Histogram of posterior distribution of mu}
\centerfigcap{0.8}{../figures/HMC_plt_trace_program_1_d_0}{Sample trace plots of mu}
\centerfigcap{0.8}{../figures/HMC_plt_log_joint_program_1}{Joint log likelihood}
\paragraph{Task 2}

\begin{minipage}{.5\textwidth}
  \centering
  \centerfigcap{1}{../figures/HMC_plt_hist_program_2_d_0}{Histogram of posterior distribution of slope}
\end{minipage}%
\begin{minipage}{.5\textwidth}
  \centering
  \centerfigcap{1}{../figures/HMC_plt_hist_program_2_d_1}{Histogram of posterior distribution of bias}
\end{minipage}

\begin{minipage}{.5\textwidth}
  \centering
  \centerfigcap{1}{../figures/HMC_plt_trace_program_2_d_0}{Sample trace plots of slope}
\end{minipage}%
\begin{minipage}{.5\textwidth}
  \centering
  \centerfigcap{1}{../figures/HMC_plt_trace_program_2_d_1}{Sample trace plots of bias}
\end{minipage}

\centerfigcap{1}{../figures/HMC_plt_log_joint_program_2}{Joint log likelihood}


\paragraph{Task 5}

\begin{minipage}{.5\textwidth}
  \centering
\centerfigcap{1}{../figures/HMC_plt_hist_program_5_d_0}{Histogram of posterior distribution of x}
\end{minipage}%
\begin{minipage}{.5\textwidth}
  \centering
\centerfigcap{1}{../figures/HMC_plt_hist_program_5_d_1}{Histogram of posterior distribution of y}
\end{minipage}

\begin{minipage}{.5\textwidth}
  \centering
  \centerfigcap{1}{../figures/HMC_plt_trace_program_5_d_0}{Sample trace plots of slope}
\end{minipage}%
\begin{minipage}{.5\textwidth}
  \centering
  \centerfigcap{1}{../figures/HMC_plt_trace_program_5_d_1}{Sample trace plots of bias}
\end{minipage}

\centerfigcap{1}{../figures/HMC_plt_log_joint_program_5}{Joint log likelihood}

\end{document}
%