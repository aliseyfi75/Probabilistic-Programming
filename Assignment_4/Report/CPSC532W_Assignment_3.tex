\documentclass{article}

\usepackage{fullpage}
\usepackage{titlesec}
\usepackage{color}
\usepackage{amsmath}
\usepackage{url}
\usepackage{verbatim}
\usepackage{graphicx}
\usepackage{parskip}
\usepackage{amssymb}
\usepackage{nicefrac}
\usepackage{listings} % For displaying code
\usepackage{algorithm2e} % pseudo-code
\usepackage{mathtools}
% for inline image 
\usepackage{float}
\usepackage{array}
\newcolumntype{P}[1]{>{\centering\arraybackslash}p{#1}}
% Python
\usepackage{xcolor}
\definecolor{codegreen}{rgb}{0,0.6,0}
\definecolor{codegray}{rgb}{0.5,0.5,0.5}
\definecolor{codepurple}{rgb}{0.58,0,0.82}
\definecolor{backcolour}{rgb}{0.95,0.95,0.92}

\setcounter{secnumdepth}{4}

\titleformat{\paragraph}
{\normalfont\normalsize\bfseries}{\theparagraph}{1em}{}
\titlespacing*{\paragraph}
{0pt}{3.25ex plus 1ex minus .2ex}{1.5ex plus .2ex}

\lstdefinestyle{mystyle}{
    backgroundcolor=\color{backcolour},   
    commentstyle=\color{codegreen},
    keywordstyle=\color{magenta},
    numberstyle=\tiny\color{codegray},
    stringstyle=\color{codepurple},
    basicstyle=\ttfamily\footnotesize,
    breakatwhitespace=false,         
    breaklines=true,    
    frame = lines,             
    captionpos=b,                    
    keepspaces=true,                 
    numbers=left,                    
    numbersep=5pt,                  
    showspaces=false,                
    showstringspaces=false,
    showtabs=false,                  
    tabsize=2
}

\lstset{style=mystyle}

\def\rubric#1{\gre{Rubric: \{#1\}}}{}

% Answers

\def\ans#1{\par\gre{Answer: #1}}

% Colors
\definecolor{blu}{rgb}{0,0,1}
\def\blu#1{{\color{blu}#1}}
\definecolor{gre}{rgb}{0,.5,0}
\def\gre#1{{\color{gre}#1}}
\definecolor{red}{rgb}{1,0,0}
\def\red#1{{\color{red}#1}}
\def\norm#1{\|#1\|}

% Math
\def\R{\mathbb{R}}
\def\argmax{\mathop{\rm arg\,max}}
\def\argmin{\mathop{\rm arg\,min}}
\newcommand{\mat}[1]{\begin{bmatrix}#1\end{bmatrix}}
\newcommand{\alignStar}[1]{\begin{align*}#1\end{align*}}
\def\half{\frac 1 2}

% LaTeX
\newcommand{\fig}[2]{\includegraphics[width=#1\textwidth]{#2}}
\newcommand{\centerfig}[2]{\begin{center}\includegraphics[width=#1\textwidth]{#2}\end{center}}
\newcommand{\centerfigcap}[3]{\begin{figure}[H]
\begin{center}\includegraphics[width=#1\textwidth]{#2} \caption{#3}\end{center}
\end{figure}}
\newcommand{\matCode}[1]{\lstinputlisting[language=Matlab]{a2f/#1.m}}
\def\items#1{\begin{itemize}#1\end{itemize}}
\def\enum#1{\begin{enumerate}#1\end{enumerate}}

\begin{document}


\title{\vspace{-20mm}
CPSC 532W Assignment 4}
\author{Ali Seyfi - 97446637}
\date{}
\maketitle

Here is the link to the repository:\\
\url{https://github.com/aliseyfi75/Probabilistic-Programming/tree/master/Assignment_4}


\blu{write a black-box variational inference (VI) evaluator following section 4.4 of the book.  }

\section{Code}
\blu{Show code snippets that demonstrate the completeness and correctness of your BBVI implementation.}
\subsection{primitives}
\label{primitives}
\lstinputlisting[language=Python, caption=primitives.py - Base primitives, firstline=145, lastline=180]{../primitives.py}
\pagebreak
\lstinputlisting[language=Python, caption=primitives.py - Functions, firstline=90, lastline=142]{../primitives.py}
\pagebreak
\lstinputlisting[language=Python, caption=primitives.py - Distributions, firstline=5, lastline=87]{../primitives.py}
\lstinputlisting[language=Python, caption=primitives.py - distlist, firstline=182, lastline=194]{../primitives.py}
\subsection{evaluation based sampling}
This part is same as last assignment.
\subsection{topological sort}
This part is same as last assignment.
\pagebreak
\subsection{BBVI evaluator}
\lstinputlisting[language=Python, caption=graph\_based\_sampling.py - BBVI\_evaluator, firstline=266, lastline=298]{../graph_based_sampling.py}
\subsection{grad log prob}
\lstinputlisting[language=Python, caption=graph\_based\_sampling.py - grad\_log\_prob, firstline=300, lastline=307]{../graph_based_sampling.py}
\pagebreak
\subsection{BBVI}
\lstinputlisting[language=Python, caption=graph\_based\_sampling.py - BBVI, firstline=309, lastline=347]{../graph_based_sampling.py}
\subsection{infinity skipper}
\lstinputlisting[language=Python, caption=graph\_based\_sampling.py - inf\_skipper, firstline=349, lastline=359]{../graph_based_sampling.py}
\pagebreak
\subsection{ELBO gradient}
\lstinputlisting[language=Python, caption=graph\_based\_sampling.py - ELBO\_gradient, firstline=361, lastline=410]{../graph_based_sampling.py}


\section{Results}
\subsection{Task 1}
$T = 10^4$ and $L = 50$ for this task.\\
Time of drawing samples: \textbf{410.81 seconds}\\
Posterior expected value of mu is: \textbf{7.3007}\\
Parameters of the posterior distribution of mu: $mu = 7.2742$ and $\sigma = 0.4931$.\\

\centerfigcap{0.6}{../figures/BBVI_plt_hist_program_1_d_0}{Histogram of posterior distribution of mu}
\centerfigcap{1}{../figures/ELBO_1}{ELBO in task 1}

\subsection{Task 2}
$T = 5*10^3$ and $L = 50$ for this task.\\
Time of drawing samples: \textbf{510.36 seconds}\\
Posterior mean of slope is: \textbf{2.1169}\\
Posterior mean of bias is: \textbf{-0.4039}\\
\centerfigcap{1}{../figures/ELBO_2}{ELBO in task 2}

\subsection{Task 3}
$T = 2*10^3$ and $L = 50$ for this task.\\
Time of drawing samples: \textbf{581.53 seconds}\\
Posterior mean of probability that the first and second datapoint are in the same cluster is: \textbf{0.7743}\\
\gre{Here we have symmetric between our states due to our choice of prior, so our optimization model will converge to different solutions each time we run the program,  which is actually the mode-seeking behaviour.}
\centerfigcap{0.7}{../figures/ELBO_3}{ELBO in task 3}
\subsection{Task 4}
Time of drawing samples: \textbf{595.87 seconds}\\
\centerfigcap{0.6}{../figures/BBVI_plt_hitmap_program_4_b0}{Posterior distribution of $b_0$}
\centerfigcap{0.6}{../figures/BBVI_plt_hitmap_program_4_w0}{Posterior distribution of $w_0$}
\centerfigcap{0.6}{../figures/BBVI_plt_hitmap_program_4_b1}{Posterior distribution of $b_1$}
\centerfigcap{0.6}{../figures/BBVI_plt_hitmap_program_4_w1}{Posterior distribution of $w_1$}
\centerfigcap{1}{../figures/ELBO_4}{ELBO in task 4}
\gre{In BBVI method, we need proposal distributions to be differentiable.  However, in parameter estimation via gradient descent we need both proposal distribution and joint distribution to be differentiable.  So the advantage of BBVI will be working with discrete and continuous models, but the variance could be a problem in this case.  But in the other method we have generally a better behaved variance.}


\subsection{Task 5}
$T = 2*10^3$ and $L = 25$ for this task.\\
Time of drawing samples: \textbf{45.19 seconds}\\
learned variational distribution for s: \textbf{Uniform(0.9024, 1.7703)} 
\centerfigcap{1}{../figures/ELBO_5}{ELBO in task 5}
\end{document}