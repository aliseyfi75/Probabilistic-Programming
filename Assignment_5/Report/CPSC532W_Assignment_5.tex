\documentclass{article}

\usepackage{fullpage}
\usepackage{titlesec}
\usepackage{color}
\usepackage{amsmath}
\usepackage{url}
\usepackage{verbatim}
\usepackage{graphicx}
\usepackage{parskip}
\usepackage{amssymb}
\usepackage{nicefrac}
\usepackage{listings} % For displaying code
\usepackage{algorithm2e} % pseudo-code
\usepackage{mathtools}
% for inline image 
\usepackage{float}
\usepackage{array}
\newcolumntype{P}[1]{>{\centering\arraybackslash}p{#1}}
% Python
\usepackage{xcolor}
\definecolor{codegreen}{rgb}{0,0.6,0}
\definecolor{codegray}{rgb}{0.5,0.5,0.5}
\definecolor{codepurple}{rgb}{0.58,0,0.82}
\definecolor{backcolour}{rgb}{0.95,0.95,0.92}

\setcounter{secnumdepth}{4}

\titleformat{\paragraph}
{\normalfont\normalsize\bfseries}{\theparagraph}{1em}{}
\titlespacing*{\paragraph}
{0pt}{3.25ex plus 1ex minus .2ex}{1.5ex plus .2ex}

\lstdefinestyle{mystyle}{
    backgroundcolor=\color{backcolour},   
    commentstyle=\color{codegreen},
    keywordstyle=\color{magenta},
    numberstyle=\tiny\color{codegray},
    stringstyle=\color{codepurple},
    basicstyle=\ttfamily\footnotesize,
    breakatwhitespace=false,         
    breaklines=true,    
    frame = lines,             
    captionpos=b,                    
    keepspaces=true,                 
    numbers=left,                    
    numbersep=5pt,                  
    showspaces=false,                
    showstringspaces=false,
    showtabs=false,                  
    tabsize=2
}

\lstset{style=mystyle}

\def\rubric#1{\gre{Rubric: \{#1\}}}{}

% Answers

\def\ans#1{\par\gre{Answer: #1}}

% Colors
\definecolor{blu}{rgb}{0,0,1}
\def\blu#1{{\color{blu}#1}}
\definecolor{gre}{rgb}{0,.5,0}
\def\gre#1{{\color{gre}#1}}
\definecolor{red}{rgb}{1,0,0}
\def\red#1{{\color{red}#1}}
\def\norm#1{\|#1\|}

% Math
\def\R{\mathbb{R}}
\def\argmax{\mathop{\rm arg\,max}}
\def\argmin{\mathop{\rm arg\,min}}
\newcommand{\mat}[1]{\begin{bmatrix}#1\end{bmatrix}}
\newcommand{\alignStar}[1]{\begin{align*}#1\end{align*}}
\def\half{\frac 1 2}

% LaTeX
\newcommand{\fig}[2]{\includegraphics[width=#1\textwidth]{#2}}
\newcommand{\centerfig}[2]{\begin{center}\includegraphics[width=#1\textwidth]{#2}\end{center}}
\newcommand{\centerfigcap}[3]{\begin{figure}[H]
\begin{center}\includegraphics[width=#1\textwidth]{#2} \caption{#3}\end{center}
\end{figure}}
\newcommand{\matCode}[1]{\lstinputlisting[language=Matlab]{a2f/#1.m}}
\def\items#1{\begin{itemize}#1\end{itemize}}
\def\enum#1{\begin{enumerate}#1\end{enumerate}}

\begin{document}


\title{\vspace{-20mm}
CPSC 532W Assignment 5}
\author{Ali Seyfi - 97446637}
\date{}
\maketitle

Here is the link to the repository:\\
\url{https://github.com/aliseyfi75/Probabilistic-Programming/tree/master/Assignment_5}

\section{Code}
\blu{Report successful in passing all of the deterministic tests in the HW support code.}

\subsection{primitives}
\blu{Provide code snippets that document critical aspects of your implementation sufficient to allow us to quickly determine whether or not you individually completed the assignment.}
\lstinputlisting[language=Python, caption=primitives.py - Env and Procedure definitions, firstline=7, lastline=27]{../primitives.py}
\lstinputlisting[language=Python, caption=primitives.py -distributions, firstline=29, lastline=86]{../primitives.py}
\lstinputlisting[language=Python, caption=primitives.py - functions, firstline=88, lastline=166]{../primitives.py}
\lstinputlisting[language=Python, caption=primitives.py - environment, firstline=169, lastline=213]{../primitives.py}

\subsection{evaluator}

\lstinputlisting[language=Python, caption=evaluator.py - standard env,  firstline=16, lastline=21]{../evaluator.py}
\lstinputlisting[language=Python, caption=evaluator.py - evaluate,  firstline=25, lastline=33]{../evaluator.py}
\lstinputlisting[language=Python, caption=evaluator.py - eval,  firstline=35, lastline=91]{../evaluator.py}

\pagebreak
\section{Program 1}
\blu{Results of Tests}\\
Here are the results of the test files:\\
\subsection{Deterministic}
tensor(7.)\\
FOPPL Tests passed\\
tensor(1.4142)\\
FOPPL Tests passed\\
tensor(24.)\\
FOPPL Tests passed\\
tensor(0.2500)\\
FOPPL Tests passed\\
tensor(0.1802)\\
FOPPL Tests passed\\
tensor([2., 3., 4., 5.])\\
FOPPL Tests passed\\
tensor(4.)\\
FOPPL Tests passed\\
tensor([2., 3., 3., 5.])\\
FOPPL Tests passed\\
tensor(2.)\\
FOPPL Tests passed\\
tensor(5.)\\
FOPPL Tests passed\\
tensor([2.0000, 3.0000, 4.0000, 5.0000, 3.1400])\\
FOPPL Tests passed\\
tensor(5.3000)\\
FOPPL Tests passed\\
\{1.0: tensor(3.2000), 6.0: tensor(2.)\}\\
FOPPL Tests passed\\
\pagebreak
\subsection{HOPPL Deterministic}
tensor(71.)\\
Test passed\\
tensor(89.)\\
Test passed\\
tensor(6.)\\
Test passed\\
tensor(1.)\\
Test passed\\
tensor([10.,  9.])\\
Test passed\\
tensor(1.)\\
Test passed\\
tensor(4.)\\
Test passed\\
tensor([11.,  2.,  8.])\\
Test passed\\
tensor(4.)\\
Test passed\\
tensor(120.)\\
Test passed\\
tensor(6.)\\
Test passed\\
tensor([2., 3., 4.])\\
Test passed\\
All deterministic tests passed\\

\subsection{Probabilistic}
('normal', 5, 1.4142136)\\
p value 0.976905685570249\\
('beta', 2.0, 5.0)\\
p value 0.9649610549178714\\
('exponential', 0.0, 5.0)\\
p value 0.28116937738711634\\
('normal', 5.3, 3.2)\\
p value 0.21365199633322285\\
('normalmix', 0.1, -1, 0.3, 0.9, 1, 0.3)\\
p value 0.718135587890695\\
('normal', 0, 1.44)\\
p value 0.6593599766116442\\
All probabilistic tests passed\\

\pagebreak
\section{Program 2}
I draw \textbf{20000} samples.\\
Mean of until success is: \textbf{98.8752}\\
Variance of until success is: \textbf{10014.3056}\\
Running time: \textbf{2 min and 9.8 seconds}

\centerfigcap{1}{../Results/fig2}{Histogram of untill success}

\pagebreak
\section{Program 3}
I draw \textbf{100000} samples.\\
Mean of $\mu$ is: \textbf{1.0016}.\\
Variance of $\mu$ is: \textbf{4.9907}\\
Running time: \textbf{38.8 seconds}

\centerfigcap{1}{../Results/fig3}{Histogram of $\mu$}

\pagebreak
\section{Program 4}
I draw \textbf{100000} samples.\\
Running time: \textbf{3 minutes and 39.4 seconds}

The distribution over states in each step is:

\[ \begin{bmatrix}
0.33351 & 0.33351 & 0.33141\\
0.15164 & 0.28333 & 0.56503\\
0.15587 & 0.21700 & 0.62713\\
0.15293 & 0.21487 & 0.63220\\
0.15289 & 0.21514 & 0.63197\\
0.15137 & 0.21454 & 0.63409\\
0.15278 & 0.21604 & 0.63118\\
0.15353 & 0.21461 & 0.63186\\
0.15249 & 0.21352 & 0.63399\\
0.15324 & 0.21412 & 0.63264\\
0.15311 & 0.21334 & 0.63355\\
0.15281 & 0.21573 & 0.63146\\
0.15356 & 0.21176 & 0.63468\\
0.15296 & 0.21367 & 0.63337\\
0.15222 & 0.21193 & 0.63585\\
0.15261 & 0.21384 & 0.63355\\
0.15254 & 0.21318 & 0.63428
\end{bmatrix} \]

The mean of the state value in each step is:

\[
\begin{bmatrix}
0.99790\\
1.41339\\
1.47126\\
1.47927\\
1.47908\\
1.48272\\
1.47840\\
1.47833\\
1.48150\\
1.47940\\
1.48044\\
1.47865\\
1.48112\\
1.48041\\
1.48363\\
1.48094\\
1.48174
\end{bmatrix}
\]

\pagebreak
The variance of the state value in each step is:

\[
\begin{bmatrix}
0.664922\\
0.545784\\
0.560920\\
0.555436\\
0.555348\\
0.552447\\
0.555099\\
0.556596\\
0.554643\\
0.556061\\
0.555843\\
0.555170\\
0.556769\\
0.555542\\
0.554178\\
0.554862\\
0.554752
\end{bmatrix}
\]
\centerfigcap{1}{../Results/fig4}{Histogram of states in each step}

\end{document}
%